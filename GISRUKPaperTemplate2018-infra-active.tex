\documentclass[11pt]{article}
\usepackage[left=25mm, right=25mm, top=25mm, bottom=25mm, includehead=true, includefoot=true]{geometry}

\usepackage{graphicx}
\usepackage{url}
\usepackage{natbib} % For referencing
\usepackage{authblk} % For author lists
\usepackage[parfill]{parskip} % Line between paragraphs

\pagenumbering{gobble} % Turn off page numbers

% Make all headings the same size (11pt):
\usepackage{sectsty}
\sectionfont{\normalsize}
\subsectionfont{\normalsize}
\subsubsectionfont{\normalsize}
\paragraphfont{\normalsize}

\usepackage{caption}
\captionsetup[table]{skip=10pt}

\renewcommand{\abstractname}{Summary} % Make 'abstract' be called 'Summary'


% This makes links and bookmarks in the pdf output (should be last usepackage command because it overrides lots of other commands)
\usepackage[pdftex]{hyperref}
\hypersetup{pdfborder={0 0 0} } % This turns off the stupid colourful border around links



% **************  TITLE AND AUTHOR INFORMATION **************

\title{Using Geocomputation to explore the active travel impacts of new roads and railways}

\author[1]{Robin Lovelace\thanks{R.Lovelace@leeds.ac.uk}}
\author[1]{Malcolm Morgan\thanks{M.Morgan1@leeds.ac.uk}}
\affil[1]{Institute for Transport Studies, University or Leeds}

\renewcommand\Authands{ and } % correct last comma in author list

\begin{document}
\date{}
\maketitle

% **************  ABSTRACT/SUMMARY  **************

\begin{abstract}
\centering


There has been relatively little GIS work focussing
on active travel impacts of infrastructure projects despite the high potential for new
schemes to affect walking and cycling, the clear geographic footprint of major projects and the increased policy interest
in these modes of transport in recent years. Motivated by the need to
ensure active travel is accounted for, this paper sets out methods for
active travel impact assessment, based on three types of impact.
We look at \emph{Severance}, when new infrastructure cuts across routes with high
active travel potential; \emph{parallels}, opportunities for
constructing new routes parallel new infrastructure; and
\emph{integration} with existing transport services, where new or
different active travel options are unlocked by new infrastructure.
The impacts are explored using example datasets and new functions developed in the
\textbf{stplanr} R package; a future direction of travel
will be applying these methods to a real case study.

$ $ \\ {\bf KEYWORDS:} Transport, Geocomputation, Active Travel, Route Analysis, Software.

\end{abstract}

\section{Introduction}

``Major transport projects may promote or discourage physical activity
in the form of walking and cycling'', yet there has been very little
quantitative \emph{a posteriori} evaluation of past projects (Ogilvie et
al. 2006), let alone \emph{a priori} assessment of potential impacts.
The present paper seeks to address this research and methodological gap,
via geographical methods and emphasis on cycling potential.

\section{A typology of active transport impacts of major transport
infrastructure
projects}\label{a-typology-of-active-transport-impacts-of-major-transport-infrastructure-projects}

Major infrastructure can impact upon active transport in a range of
different ways. To organise the assessment process, these can be
categorised into three broad types of impact.

\subsection{Severance}\label{severance}

Linear infrastructure can become a barrier to travel perpendicular to
the new infrastructure, severing routes that were used before the new
infrastructure was built. This is especially true for railways and
motorways, which may be at a different level to the surrounding road
network, and are not crossable by walkers and cyclists without dedicated
crossings. Understanding where severance occurs and how many people are
affected can aid in effectively planning the construction of bridges,
tunnels, and other crossings.

\subsection{Parallels}\label{parallels}

New linear infrastructure provides an opportunity to construct new
footpaths and cycle paths alongside the infrastructure. The marginal
cost of adding active transport routes alongside other planned
infrastructure can be lower than constructing a dedicated route. There
are several types of active transport that may take place along a
parallel route:

\begin{enumerate}
\item
  People who would have used the new infrastructure choose the active
  travel option instead;
\item
  People starting or finishing a journey at a place that is near the
  linear infrastructure but does not have access to the infrastructure
  (due to a lack of station or junction). Use active travel for part of
  their journey to reach the nearest junction/station.
\item
  People taking journeys that are shorter than the gap between
  junctions/stations
\end{enumerate}

The number of people who are affected by these scenarios will depend on
the type of infrastructure, its design, and the nature of the
surrounding area. By analysing parallels, it can aid in understanding if
an active travel route would be useful alongside on some or all of the
new linear infrastructure.

\subsection{Integration of active travel and public
transport}\label{integration-of-active-travel-and-public-transport}

New infrastructure can result in an increase in active travel in nearby
areas along the route. For example, a railway extension may result in
more people walking or cycling in the centre of town to the local
station which previously drove out of town to their destination. This
effect can manifest at a substantial distance from the new
infrastructure, by linking previously disconnected areas.

\section{Methodology}\label{methodology}

A method for assessing each of the three types of active travel impact
is outlined in this section in the abstract, such that it could be
applied to any linear feature of interest. Then Section 4 provides
specific examples using a case study of the Lewes-Uckfield train line.
The methods have been implemented as functions in the R package
\textbf{stplanr} (Lovelace and Ellison 2016).

\subsection{\texorpdfstring{Potential cycling uptake along
`parallels'}{Potential cycling uptake along parallels}}\label{potential-cycling-uptake-along-parallels}

To identify parallels to linear features, a five stage methodology was
developed.

\begin{itemize}
\item
  Subset cycling desire lines
  to include only those in which the \emph{centre point of the line}
  passes close to the new infrastructure. This was set as 10 km for this
  paper.
\item
  Break the linear infrastructure of interest into segments of even
  distance.
\item
  Calculate the angle of the segments and desire lines.
\item
  Subset the desire lines again, to include only those that are within a
  threshold angle (set at 30 degrees) of the mean bearing of their
  closest rail line segment.
\item
  Aggregate the cycling potential of all parallel lines within the
  threshold distance of each segment and assign the values to the route
  segments.
\end{itemize}

\subsection{Potential for severance}\label{potential-for-severance}

Severance occurs when active travel along cyclable desire lines with
high cycling potential is prevented or made more difficult by obstacles
such as rivers, fast roads or railway tracks. A degree of severance can
be expected along the full length of linear features, as most desire
lines have at least some cycling potential. However, the degree of
severance will be greatest in certain \emph{severance pinch points}
(henceforth referred to as severance points), for example, segments
along a rail track which intersects desire lines connecting residential
areas with employment zones on the other side.

The methodology used to identify the proposed route segments with
highest potential to cause severance (assuming the transport
infrastructure is new) consists of three stages:

\begin{itemize}
\item
  Identify the desire lines which intersect with the infrastructure.
\item
  Quantify the number of potential cyclists blocked along segments of
  the infrastructure of even length.
\item
  Subset the segments which block the highest number of potential
  cyclists and identify potential crossing points.
\end{itemize}

Note that based on the three-stage methodology outlined above, severance
points are in fact more precisely described as `severance segments',
along which a range of points could be chosen for crossing points.
However, the term `severance point' is more intuitive, so we use this
term throughout.

\subsection{Potential for cycling and public transport
integration}\label{potential-for-cycling-and-public-transport-integration}

Increased travel to public transport is likely to occur near public
transport stops and on desire lines that have an origin and destination
near connected public transport nodes (typically bus stops or train
stations). The methodology used to estimate the potenial uptake of
cycling due to integration with new public transport services associated
with the new route assumes that new nodes have suitable provision for
cycling, with safe routes to get there and sufficient cycle parking
spaces, and consists of the following stages:

\begin{itemize}
\item
  Subset trips that could feasibly use public transport in terms of
  boarding and alighting, by subsetting those with both origin and
  destination within a given distance a planned station/stop.
\item
  Remove lines that have the same stop as their closest origin and
  destination.
\item
  Remove lines which are linked by existing public transport, thus will
  not change with the new linear infrastructure.
\item
  Create new desire lines and routes that are from Origin/Destinations
  to public transport.
\item
  Remove desire lines and routes where the cycling to and from public
  transport is greater than 1.1 times the direct cycling route, as these
  travellers are unlikely to use public transport.
\item
  Calculate change in cycling potential along these new desire lines.
\end{itemize}

\section{Implementation}\label{a-case-study-of-lewes-uckfield}

The Lewes-Uckfield train line is a proposed project to restore a rail
link between Uckfield and Lewes. This is part of wider plans to increase
rail capacity in the area. As stated in the project brief, ``Such
infrastructure has the potential to support cycling if high-quality
infrastructure is built alongside.'' This appendix analyses cycling
potential along the proposed route, which is 16 km in length.

Data on existing cycling levels was
taken from the Propensity to Cycle Tool (Lovelace et al. 2017).
The data represents commuter travel between zones created for the 2011
census. These zones are known as Middle layer Super Ouput Areas (MSOA),
and the population weighted centroids of each MSOA are used as the
origin and destinations in this analysis.

There are 24 MSOAs whose population weighted centroids lie within a 10
km buffer of the train line, representing 7756 commuters.
This case study was used to illustrate each element of the proposed active travel assessment
methodology using 'geocompuational' functions written in R.

\subsection{Cycling potential parallel to the
route}\label{cycling-potential-parallel-to-the-route}

\subsubsection{Subsetting desire lines by centre point
proximity}\label{subsetting-desire-lines-by-centre-point-proximity}

The centre points of the desire lines intersecting with the
Lewes-Uckfield route buffer are presented in Figure
\ref{fig:metafigure}a. These represent 65.5\% of the desire lines in the
study area, defined by the 10 km buffer.

\begin{figure}

{\centering \includegraphics[width=0.8\linewidth]{figures/infra-selection-all}

}

\caption{The centre point-buffer (a), parallel (b), perpendicular (c) and station access (d) methods of subsetting desire lines affected by the new rail line. In all figures, the subsetted lines are plotted in red. In (b) the updated parallel desire lines, whose centre points lie within 5 km of the route, are illustrated in orange.}\label{fig:metafigure}
\end{figure}

\subsubsection{Break the train line into
segments}\label{break-the-train-line-into-segments}

Because the Lewes-Uckfield line is short and relatively straight, we
used the entire line as a single segment. However, if the line were to
be broken into segments, the results would resemble Figure
\ref{fig:segs}a. This demonstration of the method could be applied to
larger and more complex routes, e.g.~that of the proposed HS2 or HS2
cycle network route.

\begin{figure}

{\centering \includegraphics[width=0.8\linewidth]{figures/route-sel}

}

\caption{Illustration of the method of splitting the route into discrete segments using the line segment function from the stplanr R package (a) and cycling potential (under the Government Target scenario) severed by the proposed rail line (b), in which line width is proportional to the square root of cycling potential severed.}\label{fig:segs}
\end{figure}

\subsubsection{Calculate the angle of the train
routes}\label{calculate-the-angle-of-the-train-routes}

The angle of the train track was calculated by a function
\texttt{line\_bearing()} which was developed for the \textbf{stplanr} R
package specifically to solve this problem.

Using this function, the angle of the route was calculated as follows:

\begin{verbatim}
line_bearing(lewes_uckfield)
## [1] 25.30456
\end{verbatim}

\subsubsection{Subsetting desire lines parallel with the train
line}\label{subsetting-desire-lines-parallel-with-the-train-line}

To find the lines that were close to parallel with the train line, the
function \texttt{angle\_diff()} was developed. All lines within 30
degrees, clockwise or anti-clockwise, to the train line, \emph{and} have
their midpoint within the route buffer, are illustrated in the red lines
in Figure \ref{fig:metafigure}b. It is clear from this Figure that a
high number of lines were selected which are very unlikely benefit from
cycling provision along the route, especially in the south-west segment
of the plot.

To resolve this issue, a smaller buffer was used to select line centre
points. This was set at 5 km. To remove desire lines that were still far
from the train line a further subsetting method was developed. This
involved selecting desire lines that pass within an even shorter
distance to the train line, 2 km in this case. The results are presented
in Figure Figure \ref{fig:metafigure}b, in which the orange lines were
included through the centre point selection method but omitted by
`buffer intersection' method.

From the subset of the lines highlighted in red in Figure
\ref{fig:metafigure}b, we can now report summary statistics on the
cycling potential of commuter desire lines which run parallel to the
route. These results are presented alongside the
equivalent statistics for \emph{all} desire lines which intersect the 10
km buffer surrounding the proposed route.

\subsection{The potential for severance along the proposed
route}\label{the-potential-for-severance-along-the-proposed-route}

\subsubsection{\texorpdfstring{Subsetting `perpendicular' desire
lines}{Subsetting perpendicular desire lines}}\label{subsetting-perpendicular-desire-lines}

The subsetting process involved finding which lines ran perpendicular to
the proposed rail line and then selecting only those intersecting with
it, as illustrated in Figure \ref{fig:metafigure}c.

\subsubsection{Quantifying severance per segments of the train
line}\label{quantifying-severance-per-segments-of-the-train-line}

The cycling potential of the intersecting `perpendicular' lines was then
summed \emph{per 1 km segment} of the rail line. The results, for the
Government Target scenario, are presented in \ref{fig:segs}b which
shows, as one would expect, that severance impacts would be greatest at
the ends of the proposed route, where population densities and
employment opportunities are greatest.

\subsubsection{Identifying potential crossing
points}\label{identifying-potential-crossing-points}

From Figure \ref{fig:segs}b it is clear that the points of highest
potential severance lie at either end of the line. Overall, because the
proposed line does not separate any large settlements or workplaces, the
potential for severance is low. However, to demonstrate the method of
identifying places to intervene to minimise severance, Figure
\ref{fig:perpmost} illustrates the 1 km segment of the proposed line
with the highest potential for severance in context. This is clearly in
a populated part of Lewes, where travel between both sides of the new
line could be affected by the route.

\begin{figure}

{\centering \includegraphics[width=0.8\linewidth]{infra-active_files/figure-latex/perpmost-1}

}

\caption{The 1 km segment on the proposed rail line with the highest level of severance in cycling potential, under the Government Target scenario.}\label{fig:perpmost}
\end{figure}

\subsection{Cycling to Public
Transport}\label{cycling-to-public-transport}

Beyond direct impacts of the proposed scheme on cycling potential,
associated with desire lines in parallel with and crossing perpendicular
to the railway, there are indirect impacts created by the potential to
cycle to the stations(Flamm and Rivasplata 2014).
Because the proposed rail stations are located in
areas of high population density, this could generate new cycle trips
when they are taken as part of a multi-stage trip (e.g.~cycle to the
rail station, catch the train towards work, walk from the `activity end'
rail station to work).
In further work we plan to develop methods to estimate the potential uptake of cycling to public transport nodes on the system.

\section{Acknowledgements}

Thanks to the Department for Transport, the ESRC-Funded Consumer Data Research Centre and the Leeds Institute for Data Analytics.
Thanks everyone involved in the Propensity to Cycle Tool team, especially Anna Goodman, who helped develop the methods.

\section{Biography}

Robin Lovelace is a University Academic Fellow at the Leeds Institute for Transport Studies (ITS). Robin has wide ranging experience modeling sustainable transport systems and visualizing transport futures. These skills have been applied on a number of projects with real-world applications, most recently as Lead Developer of the Propensity to Cycle Tool (see www.pct.bike) and the stplanr package for sustainable transport planning. Twitter: @robinlovelace.

Malcolm Morgan is a Civil Engineer by background, but his research interests focus on Sustainable Urban Development particularly in the area of housing retrofit and transport. Malcolm is a keen Glider pilot.

% **************  REFERENCES **************

\bibliographystyle{apa}
\bibliography{documents/references.bib}

\hypertarget{ref-flamm_public_2014}{}
Flamm, Bradley J., and Charles R. Rivasplata. 2014. ``Public Transit
Catchment Areas the Curious Case of Cycle-Transit Users.''
\emph{Transportation Research Record}, no. 2419: 101--8.
doi:\href{https://doi.org/10.3141/2419-10}{10.3141/2419-10}.

\hypertarget{ref-lovelace_propensity_2017}{}
Lovelace, Robin, Anna Goodman, Rachel Aldred, Nikolai Berkoff, Ali
Abbas, and James Woodcock. 2017. ``The Propensity to Cycle Tool: An Open
Source Online System for Sustainable Transport Planning.'' \emph{Journal
of Transport and Land Use}, December.
doi:\href{https://doi.org/10.5198/jtlu.2016.862}{10.5198/jtlu.2016.862}.

\hypertarget{ref-lovelace_stplanr:_2016}{}
Lovelace, Robin, and Richard Ellison. 2016. \emph{Stplanr: Sustainable
Transport Planning}. \url{https://github.com/ropensci/stplanr}.

\hypertarget{ref-ogilvie_evaluating_2006}{}
Ogilvie, David, Richard Mitchell, Nanette Mutrie, Mark Petticrew, and
Stephen Platt. 2006. ``Evaluating Health Effects of Transport
Interventions: Methodologic Case Study.'' \emph{American Journal of
Preventive Medicine} 31 (2): 118--26.
doi:\href{https://doi.org/10.1016/j.amepre.2006.03.030}{10.1016/j.amepre.2006.03.030}.

\end{document}
